\documentclass[abstract=on, 12pt, a4paper]{scrartcl}
\usepackage[english, ngerman]{babel} 
\usepackage[T1]{fontenc}
\usepackage{microtype}
\usepackage[utf8]{inputenc}
\usepackage{bibgerm}
\usepackage{helvet}

\usepackage{bbding}  %% for checkmarks
\usepackage{color} %% for green checkmarks


\title{FoodUI Tutorial:\\How to get the project working in Android Studio}
%\subtitle{subtitle}
\author{Vadim Demkiv}
\date{\today}

\addto\captionsngerman{% 
 \renewcommand{\abstractname}{Abstract}} 


\begin{document}
\maketitle

%% define colors
 \definecolor{checkmark}{rgb}{.27,.68,.29}

%% Aufgabenbeschreibung
\abstract{
	This tutorial describes the requirements and necessary steps to get the FoodUI project working in Android Studio 2.1.2.
}


%% Requirements
\section{Requirements}
You need a few components in order to run this project:

\begin{itemize}
	\item Android Studio 2.1.2
		%TODO: Link https://developer.android.com/studio/index.html#downloads
	\item Android NDK R11c
		%TODO: Link https://developer.android.com/ndk/downloads/index.html
	\item OpenCV 3.1 (for Android)
		%TODO: Link http://opencv.org/downloads.html
	\item FoodUI project
		%TODO: Link ...
	\item Disk space for the development tools and projects: 8\,GB
		%TODO: Update
\end{itemize}


\section{Installation}
I recommend putting all the files for the android development into a mutual folder (in this tutorial I name the folder \textit{AndroidDevelopment}. In the following explanations I will refer to this folder as a root folder. 

Download the different components from the requirements, install Android Studio and extract the content from the Android NDK and OpenCV zip Files.
Keep the folder \texit{AndroidDevelopment} in mind when defining the directory for Android Studio and the Android sdk in the Android Studio installation.
Also, write down the path for the Android sdk and the extracted android ndk.

Now create a \textit{projects} folder in your \texit{AndroidDevelopment} and copy the FoodUI project into it. Also, copy the content of the folder \textit{OpenCV-android-sdk\sdk\java} from your extracted OpenCV folder into a new folder \textit{projects/OpenCV Library - 3.1.0}.




\section{Configuration}
The next step is to start Android Studio and open the FoodUI project.
Change the value of the \texttt{sdk.dir} parameter and add a \texttt{ndk.dir} parameter and value in the file \texit{local.properties} of the FoodUI project. This should be the previously mentioned. Also please note the spelling.

It could look like this:
%TODO: put code view in hier (listings stuff)
sdk.dir=C\:\\ownData\\AndroidDevelopment\\sdk
ndk.dir=C\:\\ownData\\AndroidDevelopment\\android-ndk-r11c

Now try to sync the gradles ().

If something like this comes up:
%TODO: image
click on \textit{Attempt to install all packages} in the popped up mask.

Afterwards press on the icon \textit{Sync Project with Gradle Files} to sync all gradle files.

Now open the file \texit{projects/FoodUI/app/src/main/jni/Android.mk} and change the parameter \texttt{OPENCVROOT} to the path - again, please note the spelling.
It's possible, that the folder \textit{jni} is not shown in the \textit{Android} view in Android Studio. In this case switch to the \textit{project view

It could look like this:
%TODO: put code view in hier (listings stuff)
OPENCVROOT:= C:\\ownData\\AndroidDevelopment\\OpenCV31-android-sdk

Then run the following function from the Android Studio menu in this order:
\begin{enumerate}
\item \textit{Build/Clean Project}
\item \textit{Build/Make Project}
\end{enumerate}




\section{Testing the project}
Now everything should be configured right and the project is ready to go. Eiter use your own smartphone - therefor you have to activate the developer tools and plug it in - or use the emulator within Android Studio.
I tested it on my Sony Xperia Z3 with Android 6.0.1 and it works fine.

Currently the settings in the gradles are as follows:

%TODO: put code view in hier (listings stuff)
minSdkVersion 21
targetSdkVersion 23

If you want to run this app on a lower Android version, you have to change these specifications accordingly to the desired sdk version \textb{in all gradle files}.
Though, keep in mind that the hardware requirements are tough with this implementation, so a modern smartphone is highly recommended.


\section{Final Word}
Thank you for your interest in my project and good luck with your plans.
If you have any questions, dont't hesitate to ask.


\iffalse

\fi





\end{document}